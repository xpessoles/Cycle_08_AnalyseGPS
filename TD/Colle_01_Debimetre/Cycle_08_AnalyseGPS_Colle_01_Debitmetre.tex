\documentclass[10pt,fleqn]{article} % Default font size and left-justified equations
\usepackage[%
    pdftitle={CIN : Démarche de résolution des problèmes de cinématique graphique},
    pdfauthor={Xavier Pessoles}]{hyperref}
    
\input{style/new_style}
\input{style/macros_SII}

\usepackage{multicol}
\fichetrue
%\fichefalse

\proftrue
%\proffalse

\tdtrue
%\tdfalse

\courstrue
\coursfalse

\def\discipline{Sciences \\Industrielles de \\ l'Ingénieur}
\def\xxtete{Sciences Industrielles de l'Ingénieur}

\def\classe{PTSI}
\def\xxnumpartie{Cycle 8}
\def\xxpartie{Spécification Géométrique des Produits\\
Analyser, Réaliser}

\def\xxnumchapitre{Chapitre 1}
\def\xxchapitre{Analyse des spécifications GPS}

\def\xxtitreexo{Débitmètre à turbine}
\def\xxsourceexo{\hspace{.2cm} D'après BTS IPM -- 2014.}


\def\xxposongletx{2}
\def\xxposonglettext{1.45}
\def\xxposonglety{20}
\def\xxonglet{Cycle 8 -- Ch. 1}

\def\xxactivite{Colle 1}
\def\xxauteur{\textsl{Xavier Pessoles}}

\def\xxcompetences{%
\textsl{%
\textbf{Savoirs et compétences :}\\
\noindent \textbf{Analyser :} 
\begin{itemize}[label=\ding{112},font=\color{ocre}] 
\item \textit{A3 -- C13 :} Spécifications géométriques :
\begin{itemize}
\item \textit{A3 -- C13.1 :} exigence de l'enveloppe;
\item \textit{A3 -- C13.2 :} spécification géométrique des produits;
\item \textit{A3 -- C13.3 :} tolérancement dimensionnel et géométrique.
\end{itemize}
\end{itemize}
%
%\noindent \textit{Mod2 -- C4.1 :} Représentation par schéma bloc.
}}

\def\xxfigures{
\includegraphics[width=.8\textwidth]{images/debitmetre_01}
}%figues de la page de garde

\def\xxpied{%
Cycle 8 -- Spécification Géométrique des Produits \\
Ch. 1 : Analyse des spécifications GPS -- \xxactivite%
}


\setcounter{secnumdepth}{5}
%---------------------------------------------------------------------------


\begin{document}
%\chapterimage{png/Fond_Cin}
\input{style/new_pagegarde}
\vspace{5.5cm}
\pagestyle{fancy}
\thispagestyle{plain}


\def\columnseprulecolor{\color{ocre}}
\setlength{\columnseprule}{0.4pt} 


\begin{multicols}{2}
\section*{Cours}

\subparagraph{}\textit{Recopier et compléter le tableau ci-dessous.}

\begin{center}
\includegraphics[width=.8\linewidth]{images/cours_01}
\end{center}


\section*{Exercice}

\begin{obj}~\\
\begin{itemize}
\item Justifier les spécifications d'un dessin de définition.
\item Analyser les spécifications d'un dessin de définition.
\end{itemize}
\end{obj}


L’entreprise Faure Herman située à la Ferté-Bernard conçoit, fabrique et commercialise des débitmètres pour le comptage des liquides à l’aide de compteurs à turbine ou à ultrasons pour (entre autre) l'industrie pétrolière.
\begin{center}
\includegraphics[width=\linewidth]{images/debitmetre_02}
\end{center}
Le produit support de l’étude est un débitmètre à turbine appartenant à une famille de produits comprenant 10 références de diamètre allant de 25 jusqu'à 300 mm et s’installant sur une canalisation. Le débitmètre comporte une hélice que le fluide fait tourner. Des aimants placés à sa périphérie génèrent des impulsions électriques à chaque tour, permettant ainsi le comptage de débit ou de volume. L’hélice est guidée en rotation par 2 paliers lisses.

% \renewcommand{\baselinestretch}{1.2}
%\setlength{\parskip}{2ex plus 0.5ex minus 0.2ex}


On donne dans les pages suivantes le dessin d'ensemble et une vue éclatée du débitmètre et une nomenclature ci-dessous.
\begin{center}
\includegraphics[width=\linewidth]{images/debitmetre_04}
\end{center}


On s'intéresse à l'assemblage du support de palier dont le dessin définition est donné page suivante. 
Soit les spécifications suivantes : 
\begin{center}
\begin{tabular}{cccc}
\includegraphics[width=1.5cm]{images/debitmetre_05_a} &
\includegraphics[width=1.5cm]{images/debitmetre_05_b} &
\includegraphics[width=1.5cm]{images/debitmetre_05_c} &
\includegraphics[width=1.5cm]{images/debitmetre_05_d} \\
\end{tabular}
\end{center}

\subparagraph{}\textit{Justifier les spécifications géométriques précédentes par une fonction technique que le support doit réaliser. }

\subparagraph{}\textit{Donner la signification de chacune des spécifications dimensionnelles et géométriques.}

\begin{rem}
On a : $\phi 98 f7 = \phi 98 ^{\begin{tabular}{c}-0,036 \\ -0,071\end{tabular}}$
\end{rem}
\subparagraph{}\textit{Pour garantir une bonne mise en position du support sur le débitmètre, on souhaite que le cylindre extérieur ait un défaut de cylindricité inférieur à 0,05. Indiquer cette spécification sur le dessin de définition. }
\end{multicols}

\begin{center}
\includegraphics[width=\textwidth]{images/debitmetre_07}
\end{center}


\begin{center}
\includegraphics[width=\textwidth]{images/debitmetre_06}
\end{center}
\begin{center}
\includegraphics[width=\textwidth]{images/debitmetre_03}
\end{center}

\end{document}


