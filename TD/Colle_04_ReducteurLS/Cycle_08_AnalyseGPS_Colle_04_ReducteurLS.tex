\documentclass[10pt,fleqn]{article} % Default font size and left-justified equations
\usepackage[%
    pdftitle={CIN : Démarche de résolution des problèmes de cinématique graphique},
    pdfauthor={Xavier Pessoles}]{hyperref}
    
\input{style/new_style}
\input{style/macros_SII}

\usepackage{multicol}
\fichetrue
%\fichefalse

\proftrue
%\proffalse

\tdtrue
%\tdfalse

\courstrue
\coursfalse

\def\discipline{Sciences \\Industrielles de \\ l'Ingénieur}
\def\xxtete{Sciences Industrielles de l'Ingénieur}

\def\classe{PTSI}
\def\xxnumpartie{Cycle 8}
\def\xxpartie{Spécification Géométrique des Produits\\
Analyser, Réaliser}

\def\xxnumchapitre{Chapitre 1}
\def\xxchapitre{Réducteur Leroy Somer}

\def\xxtitreexo{Débitmètre à turbine}
\def\xxsourceexo{\hspace{.2cm} D'après BTS IPM -- 2012.}


\def\xxposongletx{2}
\def\xxposonglettext{1.45}
\def\xxposonglety{20}
\def\xxonglet{Cycle 8 -- Ch. 1}

\def\xxactivite{Colle 4}
\def\xxauteur{\textsl{Xavier Pessoles}}

\def\xxcompetences{%
\textsl{%
\textbf{Savoirs et compétences :}\\
\noindent \textbf{Analyser :} 
\begin{itemize}[label=\ding{112},font=\color{ocre}] 
\item \textit{A3 -- C13 :} Spécifications géométriques :
\begin{itemize}
\item \textit{A3 -- C13.1 :} exigence de l'enveloppe;
\item \textit{A3 -- C13.2 :} spécification géométrique des produits;
\item \textit{A3 -- C13.3 :} tolérancement dimensionnel et géométrique.
\end{itemize}
\end{itemize}
%
%\noindent \textit{Mod2 -- C4.1 :} Représentation par schéma bloc.
}}

\def\xxfigures{
\includegraphics[width=.8\textwidth]{images/reducteur_01}
}%figues de la page de garde

\def\xxpied{%
Cycle 8 -- Spécification Géométrique des Produits \\
Ch. 1 : Analyse des spécifications GPS -- \xxactivite%
}


\setcounter{secnumdepth}{5}
%---------------------------------------------------------------------------


\begin{document}
%\chapterimage{png/Fond_Cin}
\input{style/new_pagegarde}
\vspace{7cm}
\pagestyle{fancy}
\thispagestyle{plain}


\def\columnseprulecolor{\color{ocre}}
\setlength{\columnseprule}{0.4pt} 

\begin{multicols}{2}

\begin{obj}~\\
\begin{itemize}
\item Justifier les spécifications d'un dessin de définition.
\item Analyser les spécifications d'un dessin de définition.
\end{itemize}
\end{obj}

Le moto réducteur de vitesse électromécanique LEROY SOMER  est un appareil à roue et vis. Il est particulièrement compact et léger tout en gardant de hautes performances.
Sa conception est modulaire et permet de nombreuses adaptations (motorisation, arbre de sortie, position de montage …) afin de répondre au mieux aux problèmes posés. 

On donne dans les pages suivantes vue éclatée du réducteur.


On s'intéresse au flasque dont le dessin définition est donné page suivante. 
Soit les spécifications suivantes : 
\begin{center}
\begin{tabular}{cccc}
\includegraphics[width=1.5cm]{images/reducteur_04_a} &
\includegraphics[width=1.5cm]{images/reducteur_04_b} &
\includegraphics[width=1.5cm]{images/reducteur_04_c} &
\includegraphics[width=1.5cm]{images/reducteur_04_d} \\
\end{tabular}
\end{center}

\subparagraph{}\textit{Justifier les spécifications géométriques précédentes par une fonction technique que le support doit réaliser. }

\subparagraph{}\textit{Donner la signification de chacune des spécifications dimensionnelles et géométriques.}

\begin{rem}
On a : $\phi 70 H6 = \phi 78 ^{\begin{tabular}{c}+19 \\ 0\end{tabular}}$
\end{rem}
\subparagraph{}\textit{Pour garantir une bonne mise en position des éléments roulants par rapport au flasque, on souhaite que le défaut de cylindricité sur les portées de roulements soit inférieur à 0,05 mm. Indiquer cette spécification sur le dessin de définition.}
\end{multicols}

\begin{center}
\includegraphics[width=\textwidth]{images/reducteur_03}
\end{center}


\begin{center}
\includegraphics[width=\textwidth]{images/reducteur_02}
\end{center}


\end{document}


